% !TEX root = ../thesis.tex

\chapter{介绍}
过去十年见证了大数据的爆炸式发展,人类社会以前所未有的规模和速度创建大量数据。第二次工业革命以后,数据量越每10年翻一番;
从20世纪50年中期开启的信息革命带来的巨大生产力发展,导致人类社会从工业化时代进入信息化时代,新数据的产生进一步加速,约每三年翻一番。
前谷歌首席执行官埃里克·施密特在Techonomy'10会议上分享,现在短短2天产生的数据量比2003年之前整个人类社会产生的数据总量还多。
据国际数据公司(IDC)发布的2017年大数据白皮书预测\cite{reinsel2017data},2025年全球大数据规模将增长至163ZB,相当于2016年的10倍。
大量数据的可利用性催生了大规模的数据密集型应用。大数据分析迅速成为一门重要研究,在金融、商业、医疗保健和政府管理等重要领域发挥作用,进而服务整个社会。
例如电商公司和广告公司可以对用户历史数据进行分析,来给用户提供个性化的产品;银行等金融机构的风控部门可以对每天的交易数据进行分析,打击金融诈骗等违法犯罪行为;
政府部门可以分析一段时间的人流数据,决定新的地铁站最合适的位置。

随着数字世界的不断扩展和大数据分析技术在社会生活中重要性的日益增长,公司、政府机构和学术机构对可快速运行大数据分析程序的可扩展系统的需求也日益增长。
解决可扩展性挑战的主流方法是使用大量计算节点来进行分布式处理。常用的大数据处理系统包括Hadoop\cite{hadoop}、Spark\cite{spark}、Hive\cite{hive}、DryadLinQ\cite{dryadLinQ}、Flink\cite{flink}等。这些大数据处理系统大多是使用Java、
C\#或者Scala等托管式语言开发的,因为托管式语言提供了快速开发的特性:自动内存管理和丰富的社区支持。但是使用托管式语言开发的程序必须要运行在托管式语言提供的
运行环境(runtime)或者更具体一点是托管式语言提供的虚拟机上。但是这些语言虚拟机在最初设计时并没有考虑到大数据的特性:数据量巨大和数据对象生命周期的epoch特性\cite{nguyen2016yak}, 因此大数据分析程序运行时,runtime
会带来巨大的性能开销,例如runtime内存管理中垃圾回收(GC)常常会占用40\% 到 60\%的执行时间。

托管式运行时带来的如此大的性能开销是不可以接受的,那我们是不是可以抛弃托管式语言,回到非托管式语言?答案是否定的。使用非托管式语言例如C或C++进行开发确实可以避免托管式语言带来的
内存管理开销,但这也丧失托管式语言带来的便利:快速开发。众所周知,使用非托管式语言进行开发需要自己做内存管理等一系列工作,开发者需要写的代码更多,相应出错的概率也就越大。并且,
以非托管式语言调试内存错误是一项很痛苦的任务,尤其是在大数据处理系统数据量大、分布式执行环境和运行时间较长的特性下,调试的工作量更大。同时,由于大多数现有的大数据处理框架已经以
托管式语言进行开发,全部推倒以非托管式语言重构工作量太大,基本是不现实的。

因此,主流的研究方向还是在对现有的托管式运行时,主要是语言虚拟机进行优化。本文对大数据场景的研究进行调研,总结高级语言虚拟机(主要是JVM和CLR)中存在的问题。
并针对其中一个研究项目Yak\cite{nguyen2016yak},具体分析它的研究思路,指出它的局限性和适用范围。最后本文针对Yak的局限性,提出了一个具体的优化方案,并分析该方案的可行性。






